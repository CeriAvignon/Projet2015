%Rapport Séance 2 - Groupe 5

\documentclass[a4paper,11pt,final]{article}

%%%%%%%%%%%%%%%%%%%%%%%%%%%%%%%%%%%%%%%%%%%%%%%%%%%
%% Packages utilisés
%%%%%%%%%%%%%%%%%%%%%%%%%%%%%%%%%%%%%%%%%%%%%%%%%%%
\usepackage[english,francais]{babel}
\usepackage[utf8]{inputenc}
\usepackage[T1]{fontenc}
\usepackage{lmodern}
\usepackage[pdftex]{graphicx}
\usepackage[top=2.5cm, bottom=2.5cm, left=2.5cm, right=2.5cm]{geometry}
\usepackage{setspace}
\usepackage[colorlinks=true]{hyperref}
\usepackage[french]{varioref}
\usepackage{lastpage}
\usepackage{fancyhdr}
\usepackage[numbers]{natbib}
\usepackage[table]{xcolor}
\usepackage{tikz}

\setlength{\headheight}{13.6pt} % due to a warning

% pour la mise en forme de l'uapv
\usepackage{soul}
\usepackage{graphicx}
\sodef\ugg{}{.4em plus 1fill}{1em plus 2 fill}{2em plus 2fill minus.1em}



%%%%%%%%%%%%%%%%%%%%%%%%%%%%%%%%%%%%%%%%%%%%%%%%%%%
%% Informations générales
%%%%%%%%%%%%%%%%%%%%%%%%%%%%%%%%%%%%%%%%%%%%%%%%%%%
%TODO Titre du document, à adapter
\newcommand{\reporttitle}{Rapport Séance 2 - Groupe 5}    

%TODO Liste des auteurs
\newcommand{\reportauthors}{
	Cyril NOVOTNY 
	Dimitri HUEBER 
} 

%TODO UE concernée par le rapport (à modifier)
% exemples : UE 
\newcommand{\classname}{Projet Ingenierie Logicielle}

% TODO formation concernée : à compléter
% exemples : Licence Informatique, Master Informatique
\newcommand{\formation}{Licence Informatique}

% TODO parcours ou spécialité de la formation
% exemples pour la licence : Parcours Systèmes et Réseaux Informatiques, Parcours Ingénierie Logicielle
% exemples pour le master : Spécialité Ingénierie du Logiciel pour la Société Numérique, Spécialité Réseaux Informatiques et Services Mobiles
\newcommand{\parcours}{Option Ingenierie Logicielle}


\newcommand{\HRule}{\rule{\linewidth}{0.5mm}}
% Espace entre les paragraphes
%\setlength{\parskip}{1ex} 
% En-tête et pied de page
\pagestyle{fancy}
\fancyhf{}
\renewcommand{\headrulewidth}{0.4pt}
\renewcommand{\footrulewidth}{0.4pt}
\cfoot{\thepage\ / \pageref*{LastPage}} 
\chead{\reporttitle} 
% couleurs
\definecolor{lightcolor}{gray}{0.7}
\definecolor{darkcolor}{gray}{0.9}
\definecolor{grisclair}{rgb}{0.96,0.96,0.96}
\definecolor{grisfonce}{rgb}{0.83,0.83,0.83}
\definecolor{vert}{rgb}{0.64,0.69,0.31}
%\definecolor{vertfonce}{rgb}{0.54,0.59,0.21}
% infos liées à la génération de PDF
\hypersetup{
    pdftitle={\reporttitle},%
    pdfauthor={\reportauthors},%
    pdfsubject={\classname},%
%	bookmarks=true,
    bookmarksnumbered=true,bookmarksopen=true,
	unicode=true,colorlinks=true,linktoc=all,%linktoc=page
	linkcolor=blue,citecolor=blue,filecolor=blue,urlcolor=blue,
	pdfstartview=FitH
}
% police de caractères
\renewcommand{\familydefault}{\sfdefault}
% listes à points
\renewcommand{\FrenchLabelItem}{\textbullet}

\begin{document}



%%%%%%%%%%%%%%%%%%%%%%%%%%%%%%%%%%%%%%%%%%%%%%%%%%%
%% Page de titre (à ne pas modifier)
%%%%%%%%%%%%%%%%%%%%%%%%%%%%%%%%%%%%%%%%%%%%%%%%%%%
	% page titre
		\phantomsection  
		\addcontentsline{toc}{section}{Titre}	% add the title page in the TOC (yes!)
		\begin{titlepage}
			\begin{tikzpicture}[remember picture,overlay]
				% vertical lines
		    	\node at (current page.south west)
				{	\begin{tikzpicture}[remember picture,overlay]
						\fill[fill=grisclair]  (0cm,21.2cm) rectangle(21cm,29.7cm);
						\fill[fill=grisfonce] (0cm,0cm) rectangle(21cm,21.2cm);
						\fill[fill=vert] (1cm,0cm) rectangle(5.2cm,21.2cm);
 						\pgftext[x=1cm,y=21.4cm,bottom,left]{\includegraphics[width=4.2cm]{images/UAPV-intitule-CMJN.png}};
 						\pgftext[x=1.1cm,y=20.5cm,bottom,left]{\scalebox{0.6}[1]{\fontsize{13}{13}{\fontfamily{ptm}\selectfont{}\makebox[6.7cm][l]{\ugg{\textbf{CENTRE}}}}}};
 						\pgftext[x=1.1cm,y=20cm,bottom,left]{\scalebox{0.6}[1]{\fontsize{13}{13}{\fontfamily{ptm}\selectfont{}\makebox[6.7cm][l]{\ugg{\textbf{D'ENSEIGNEMENT}}}}}};
 						\pgftext[x=1.1cm,y=19.5cm,bottom,left]{\scalebox{0.6}[1]{\fontsize{12}{12}{\fontfamily{ptm}\selectfont{}\makebox[6.7cm][l]{\ugg{\textbf{ET DE RECHERCHE}}}}}};
 						\pgftext[x=1.1cm,y=18.9cm,bottom,left]{\scalebox{0.6}[1]{\fontsize{12}{12}{\fontfamily{ptm}\selectfont{}\makebox[6.7cm][l]{\ugg{\textbf{EN INFORMATIQUE}}}}}};
 						\pgftext[x=5.5cm,y=18.2cm,bottom,left]{\scalebox{0.6}[1]{\fontsize{13}{13}{\fontfamily{phv}\selectfont{}\textbf{\formation}}}};
 						\pgftext[x=5.5cm,y=17.7cm,bottom,left]{\scalebox{0.6}[1]{\fontsize{13}{13}{\fontfamily{phv}\selectfont{}\textbf{\parcours}}}};
 						\pgftext[x=5.5cm,y=17.3cm,bottom,left]{\scalebox{0.6}[1]{\fontsize{13}{13}{\fontfamily{phv}\selectfont{}\textbf{UE} \classname}}};
% 						\pgftext[x=3.5cm,y=16.2cm,bottom,left]{\scalebox{0.77}[1]{\fontsize{30}{30}{\fontfamily{phv}\selectfont{}\textbf{}}}};
 						\pgftext[x=3.5cm,y=15.5cm,bottom,left]{\scalebox{0.77}[1]{\fontsize{30}{30}{\fontfamily{phv}\selectfont{}\textbf{\textcolor{white}{>{}>{}>}\hspace{0.7cm}\textcolor{vert}{\parbox{19cm}{\raggedright\reporttitle}}}}}};
 						\pgftext[x=5.5cm,y=14.5cm,bottom,left]{\scalebox{0.77}[1]{\fontsize{20}{20}{\fontfamily{phv}\selectfont{}\textcolor{vert}{\reportauthors}}}};
 						\pgftext[x=5.5cm,y=13.1cm,bottom,left]{\scalebox{0.6}[1]{\fontsize{13}{13}{\fontfamily{phv}\selectfont{}\textbf{\today}}}};
 						\pgftext[x=1.1cm,y=5.2cm,bottom,left]{\scalebox{0.6}[1]{\fontsize{12}{12}{\fontfamily{ptm}\selectfont{}\makebox[6.7cm][c]{CERI - LIA}}}};
 						\pgftext[x=1.1cm,y=4.7cm,bottom,left]{\scalebox{0.6}[1]{\fontsize{12}{12}{\fontfamily{ptm}\selectfont{}\makebox[6.7cm][c]{339 chemin des Meinajariès}}}};
 						\pgftext[x=1.1cm,y=4.4cm,bottom,left]{\scalebox{0.6}[1]{\fontsize{12}{12}{\fontfamily{ptm}\selectfont{}\makebox[6.7cm][c]{BP 1228}}}};
 						\pgftext[x=1.1cm,y=4.0cm,bottom,left]{\scalebox{0.6}[1]{\fontsize{12}{12}{\fontfamily{ptm}\selectfont{}\makebox[6.7cm][c]{84911 AVIGNON Cedex 9}}}};
 						\pgftext[x=1.1cm,y=3.6cm,bottom,left]{\scalebox{0.6}[1]{\fontsize{12}{12}{\fontfamily{ptm}\selectfont{}\makebox[6.7cm][c]{France}}}};
 						\pgftext[x=1.1cm,y=2.9cm,bottom,left]{\scalebox{0.6}[1]{\fontsize{12}{12}{\fontfamily{ptm}\selectfont{}\makebox[6.7cm][c]{Tél. +33 (0)4 90 84 35 00}}}};
 						\pgftext[x=1.1cm,y=2.4cm,bottom,left]{\scalebox{0.6}[1]{\fontsize{12}{12}{\fontfamily{ptm}\selectfont{}\makebox[6.7cm][c]{Fax +33 (0)4 90 84 35 01}}}};
 						\pgftext[x=1.1cm,y=1.8cm,bottom,left]{\scalebox{0.6}[1]{\fontsize{12}{12}{\fontfamily{ptm}\selectfont{}\makebox[6.7cm][c]{http://ceri.univ-avignon.fr}}}};                        
					\end{tikzpicture}
				};
			\end{tikzpicture}
		\end{titlepage}

		\setcounter{page}{2} 	% set the second page... to number 2
		\thispagestyle{plain}	% force header/footer
  
  	% Table des matières
	\cleardoublepage % Dans le cas du recto verso, ajoute une page blanche si besoin
	\phantomsection
	\addcontentsline{toc}{section}{Table des matières}
	\tableofcontents
	\thispagestyle{fancy}
	
	% Justification moins stricte : des mots ne dépasseront pas des paragraphes
	\sloppy          
  
  
%%%%%%%%%%%%%%%%%%%%%%%%%%%%%%%%%%%%%%%%%%%%%%%%%%%
%% Présentation de la composante
%%%%%%%%%%%%%%%%%%%%%%%%%%%%%%%%%%%%%%%%%%%%%%%%%%%
\cleardoublepage
\section{Présentation de la composante}	  
La composante en question est celle du Moteur Physique. 
Elle permet de gérer l’aspect technique d’un “round”, c’est-à-dire d’une manche. Elle touche donc aux calculs pour les déplacements et les placements de départ, aux collisions, aux bonus et malus ainsi qu’aux conditions de victoire et de défaite.

%%%%%%%%%%%%%%%%%%%%%%%%%%%%%%%%%%%%%%%%%%%%%%%%%%%
%% Interactions avec les autres composantes
%%%%%%%%%%%%%%%%%%%%%%%%%%%%%%%%%%%%%%%%%%%%%%%%%%%
\section{Interactions avec les autres composantes}	 
Le moteur physique envoie des données aux moteurs réseaux et graphiques ainsi qu’à l’interface graphique.

\subsection{Avec le Moteur Graphique} 
Pour permettre l’affichage pur et simple des actions, des calculs etc.. de façon à être représenté autrement qu’en simples lignes.

\subsection{Avec l'Interface Utilisateur} 
Pour permettre le classement des joueurs.

\subsection{Avec le Moteur Réseaux} 
Pour permettre un affichage synchronisé des actions sur les différents utilisateurs en jeu.

%%%%%%%%%%%%%%%%%%%%%%%%%%%%%%%%%%%%%%%%%%%%%%%%%%%
%% Structure des données partagées
%%%%%%%%%%%%%%%%%%%%%%%%%%%%%%%%%%%%%%%%%%%%%%%%%%%
\section{Structure des données partagées}	  
Classe Snake partagée avec MR et MG
Classe Bonus/Malus partagée avec MG et MR
Classe Field partagée avec MG et MR et IG
Getteur et setteur

%%%%%%%%%%%%%%%%%%%%%%%%%%%%%%%%%%%%%%%%%%%%%%%%%%%
%% Méthodes à implémenter et à invoquer
%%%%%%%%%%%%%%%%%%%%%%%%%%%%%%%%%%%%%%%%%%%%%%%%%%%
\section{Méthodes à implémenter et à invoquer}
Classe Player partagée par l’IG avec nous
Initialisation de la partie (ID Joueur)
MAJ des positions des objets	  

%%%%%%%%%%%%%%%%%%%%%%%%%%%%%%%%%%%%%%%%%%%%%%%%%%%
%% En-tête et fonctionnalités
%%%%%%%%%%%%%%%%%%%%%%%%%%%%%%%%%%%%%%%%%%%%%%%%%%%
\section{En-tête et fonctionnalités}

\subsection{Round}	  
	Round : Classe générale comportant le tableau des joueurs (Player[]) et le plateau de jeu (Field)

\subsection{Field}	  
	Field : Classe contenant chaque case du plateau (Case Map[L][l]) ainsi qu’une longueur, une largeur et un taux de drop.

\subsection{Case}	  
Case : Classe représentant une case du plateau. Elle contient ses coordonnées (coordx et coordy) ainsi que son possesseur qui est soit un joueur soit nul (owner) et sa présence ou non d’un bonus ou malus général ou personnel (BME).

\subsection{Snake}	  
Snake : Classe contenant toutes les informations sur le Snake : 
Position (X et Y);
Largeur;
Vitesse;
Appartenance;
Direction;
Est vivant;
Son taux de trous.

\subsection{BME}	  
BME : Classe abstraite rassemblant les différents bonus et malus du jeu, qu’ils soit uniquement pour un joueur, ou les touche tous. Ils sont donc répartis en deux autres classes : BonusMalus et Event.

\subsection{BonusMalus}	  
BonusMalus : Largeur, Vitesse, Collision, Inverse, "Gruyère"

\subsection{Event}	  
Event : Clean (efface les tracés), Thunder (réduit la largeur des autres snake), More Loops (augmente le drop de bonus/malus), Earth (ou mode avion, sortir d'un côté du terrain nous amène de l'autre côté)

\end{document}